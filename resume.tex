\documentclass[11pt,serif]{cv}

\usepackage[
    style=numeric-comp,
    sortcites=false,
    sorting=ydnt,
    defernumbers=true,
    url=false,
    maxbibnames=5
  ]{biblatex}

\bibliography{refs}

\def\Peter{Dr.~A.~Peter~Young}
\def\John{Dr.~John~Price}
\def\cpp{C\nolinebreak[4]\hspace{-.05em}\raisebox{.4ex}{\tiny\bf ++}}

\name{Matthew}{Wittmann}
\belowname{Palo Alto, CA}
\phone{+1 303 552 7753}
\email{mcwitt@gmail.com}
\website{\href{https://mcwitt.github.io}{\nolinkurl{mcwitt.github.io}}}
\linkedin{\href{https://linkedin.com/in/mcwittmann}{\nolinkurl{mcwittmann}}}
\github{\href{https://github.com/mcwitt}{\nolinkurl{github.com/mcwitt}}}
\address{Palo Alto, CA}
%\photo{me.jpg}
\photo{}

\begin{document}

\maketitle
\vskip-5em
\vfill

\section{Education}

\begin{cvjobs}
  \cvjob{2015}{}
    {PhD in Physics}
    {University of California, Santa Cruz}{}
    %{Supervised by \Peter{} (see \nameref{sec:research}).}
  \cvjob{2009}{}
    {BA in Physics and Mathematics}
    {University of Colorado, Boulder}{}
    %{\textit{Summa Cum Laude} with minor in Computer Science, 3.8 GPA. \\
    % Undergraduate thesis supervised by \John.}
\end{cvjobs}

%\subsection{Workshops \& Short Courses}
%
%\begin{cvjobs}
%  \cvjob{6/2013}{}
%    {Beg Rohu School of Statistical Physics and Condensed Matter}
%    {Quiberon, France}
%    {Two-week course with special focus on disordered systems.}
%  \cvjob{9/2012}{}
%    {Efficient Algorithms in Computational Physics}
%    {Bad Honnef, Germany}
%    {Two-week course focusing on Monte Carlo techniques. I assisted students with
%     homework problems during the data analysis portion of the course, taught by
%     my PhD advisor.} 
%\end{cvjob}

\section{Experience}
\label{sec:research}

\begin{cvjobs}
  \cvjob{2015}{}
    {Fellow}
    {Insight Data Science}
    {%
      \begin{itemize}
        \item Created
          \href{http://scenicstroll.xyz}{\texttt{scenicstroll.xyz}}, a web app
          that recommends scenic walking routes
        \item Retreived and stored $\sim 300$k geotagged photos from Flickr and
          SF street data from OpenStreetMap using Python and PostgreSQL
        \item Designed algorithm to find scenic routes using kernel density
          estimation and graph search
        \item Built frontend using Flask, Bootstrap, jQuery, and Leaflet.js; deployed on AWS
      \end{itemize}
    }
  \cvjob{2010}{2015}
    {Graduate Student Researcher}
    {University of California, Santa Cruz}
    {%
      \begin{itemize}[topsep=0pt]
        \item Probed the low-temperature phase of ``spin-glass'' models using
          Monte Carlo simulations
        \item Developed simulations in C and Python for deployment on research
          computer clusters
        \item Analyzed large ($\sim 10$ GB) data sets from simulations using
          Python, pandas, and HDF5
        \item Published 4 first-author papers on original research and one
          with international collaborators
        \item Taught 6 lower- and upper-division Physics and Math courses as a
          teaching assistant
      \end{itemize}
    }
  \cvjob{2014}{}
    {Guest Researcher}
    {Max Planck Institute, Dresden, Germany}
    {%
      \begin{itemize}
        \item
          Developed algorithms in C and Python for large-scale simulations of
          spin-glass dynamics
        \item
          Performed simulations to investigate the relationship between spin
          glass dynamics and state-of-the art static calculations
        \item
          Analyzed large datasets involving averages over thousands of models,
          each with millions of degrees of freedom
      \end{itemize}
    }
  \cvjob{2008}{2009}
    {Undergraduate Research Assistant}
    {University of Colorado, Boulder}
    {%
      \begin{itemize}
        \item
          Developed AcousticVNA, a low-cost system for acoustic network
          analysis, in collaboration with the Price group
        \item
          Validated system performance by comparison with results of
          finite-element analysis
        \item
          Developed and implemented an extended calibration technique for
          vector network measurements using MATLAB
      \end{itemize}
    }
\end{cvjobs}

%\section{Teaching \& Outreach}
%
%\begin{cvjobs}
%  \cvjob{2014}{}
%    {Juror at USA Young Physicists Tournament}{San Jose, CA}{}
%  \cvjob{}{}
%    {Judge at Pacific Collegiate School Science Fair}{Santa Cruz, CA}{}
%  \cvjob{2009}{2011}
%    {TA in Physics and Mathematics}
%    {University of California, Santa Cruz}
%    {Taught lower- and upper-division physics lab courses and led discussion sections in
%     lower-division math and upper-division physics lecture courses.}
%  \cvjob{2008}{2009}
%    {Instructor Assistant in Mathematics}
%    {University of Colorado, Boulder}
%    {Led tutorials in supplemental math courses at the precalculus level.}
%\end{cvjobs}

\section{Skills}

\begin{cvskills}
  \cvskill{Specialties}{
    Probability~and~statistics,
    predictive~modeling,
    Monte~Carlo~simulation,
    optimization,
    software~engineering%
    \footnote{\label{fn:some}familiar; interested in developing}
  }
  \cvskill{Languages}{
    Python,
    C,
    SQL,
    MATLAB,
    \LaTeX,
    HTML/CSS,
    JavaScript,
    R\footnoteref{fn:some},
    Scala\footnoteref{fn:some},
    \cpp\footnoteref{fn:some}
  }
  \cvskill{Tools}{
    NumPy,
    SciPy,
    pandas,
    scikit-learn,
    Git,
    PostgreSQL,
    jQuery
  }
\end{cvskills}

%\printbibliography[title={Publications},type=article,prefixnumbers={J}]
%\printbibliography[title={Submitted papers},type=online,prefixnumbers={U},heading=subbibliography]
%\printbibliography[title={Conferences},type=misc,prefixnumbers={C},heading=subbibliography]

\end{document}
